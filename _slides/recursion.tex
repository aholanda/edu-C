\lstset{basicstyle=\normalsize}

\title{Recursão}
\frame{\maketitle}

\begin{frame}{``Drawing hands'', M. C. Escher}
\begin{center}
  \includegraphics[scale=.725]{img/drawingHands-escher.png}
\end{center}
\end{frame}

\begin{frame}{Arte gráfica influenciada por Escher}
\begin{center}
  \includegraphics[scale=.2]{img/recursive-escher.png}
\end{center}
{\vfill\tiny Fonte: \url{http://inspirabox.com/inspiration/16-M.-C.-Escher-Recursive/}}
\end{frame}

\begin{frame}{Recursão}{Definição}
  \begin{itemize}
  \item {\bf Algoritmo recursivo}: Quando um problema pode ser
    dividido em instâncias menores, e o algoritmo usado para as
    instâncias menores é o mesmo do problema original, dizemos que
    este algoritmo é recursivo.
    \pause
  \item {\bf Função recursiva}: Uma função recursiva contem a
    implementação de um algoritmo recursivo e pode chamar a si própria,
    tendo como argumento instâncias cada vez menores da entrada, para no
    final processar toda a entrada do problema a ser solucionado.
  \end{itemize}
\end{frame}

 \begin{frame}[fragile]{Fatorial}{Exemplo}
 
 Vamos adotar a seguinte fórmula para a implementação do cálculo do fatorial:

  \begin{numcases}{n!=}
   0,  & se $n < 0$, \\
   1,  & se $n = 0$, \\
   \prod_{k=1}^n k, & se $n \geq 1$.
 \end{numcases}
 
\end{frame}

\begin{frame}[fragile]{Fatorial sem recursão}
  
  Implementação usando o método iterativo:\bigskip

  \begin{lstlisting}
    int Fatorial(int n)
    {
        int i, fat = n >= 0 ? 1 : 0;
      
        for (i = 2; i <= n; i++)
            fat = fat * i;

        return fat;
    }
  \end{lstlisting}

\end{frame}

\begin{frame}[fragile]{Fatorial com recursão}
  
  Implementação usando o método recursivo:\bigskip

  \begin{lstlisting}
    int FatorialR(int n)
    {
      if (n < 0)
          return 0;
      else if (n == 0 || n == 1)
          return 1;
      else
          return n * FatorialR(n - 1);
    }
  \end{lstlisting}
\end{frame}

\begin{frame}[fragile]{Fatorial com recursão, cont.}{Exemplo}
  Cálculo recursivo de $4!$ usando a função {\tt FatorialR()}.
  
  \bigskip
  \begin{lstlisting}
    FatorialR(4)
        |_ 4 * FatorialR(3)
                   |_ 3 * FatorialR(2)
                              |_ 2 * FatorialR(1)
                                 2 * 1  _|
                      3 * 2  _|
           4 * 6  _|
    24 _|
  \end{lstlisting}
\end{frame}

\begin{frame}{Vantagens}{Recursão}
  \begin{itemize}
  \item Pelo técnica de divisão e conquista, as instâncias menores do
    problema podem ter menos passos do que a solução que processa toda
    a instância do problema.
    \pause
  \item Antigamente, o custo das chamadas de funções era relativamente
    alto, porém, com a evolução de técnicas de otimização dos
    compiladores e processadores, este custo pode ser normalmente 
    desconsiderado.
  \end{itemize}
\end{frame}

\begin{frame}{Desvantagem}{Recursão}  
  \begin{itemize}
  \item Se o número de instâncias menores do problema for muito grande, 
    pode haver estouro da pilha ({\it stack}).
  \end{itemize}
  
\end{frame}